%-----------------------------------------------------------------------------
%
%          PHYSICS  M.S.     THESIS
%          JUSTIN A. VASEL
%
%          This began as the template offered by the University of Minnesota, 
%          but I've made a few changes here and there...  
%
%          -->  neutrinos.tex
%
%-----------------------------------------------------------------------------


\chapter{The Physics of Neutrinos}
	\label{neutrino_physics_chapter}

	\vspace{-0.2in}

	\begin{quoting}
		\noindent \large ``Neutrinos, they are very small." \normalsize

		--- John Updike
	\end{quoting}

	\chapterIntro{T}{he turn of the twentieth century} marked the beginning of a golden age in physics. The formulation of relativity and formalism of quantum mechanics brought with it a wealth of paradigm-shifting discoveries, offering humanity a deeper, more fundamental understanding of the Universe and paving the way for decades of ground-breaking research. One of the crowning achievements of this era has been the construction of the Standard Model of particle physics.

	The advent of the model was paramount to the advancement of modern physics. Through its theoretical framework, the particle physicist can calculate, among other things: decay modes, branching ratios and intrinsic properties of a myriad of particles. In some instances, the very existence of yet unobserved particles can be inferred from the Standard Model. The prediction of the Higgs boson is perhaps the most notable example, being initially theorized in 1964\cite{higgsTheory} and finally discovered in 2012\cite{higgsCMS,higgsATLAS}.

	It is essential in particle physics that one be able to predict the observed phenomena before it is observed. While atomic particles like the proton, neutron, and electron are easy to detect, many particles are not. This is especially true for particles that have a small mass or no charge, like the neutrino, or very brief lifetimes like the Higgs boson. It is nearly impossible to detect these particles directly. Instead, one must depend on indirect observations like the products of an interaction or the daughter particles of a decay sequence. To make new discoveries, the particle physicist must know what she is looking for and where she might find it; it is the nature of the beast. 

	
	%% SECTION : THE STANDARD MODEL OF PARTICLES
	\section{The Standard Model of Particles}

		The Standard Model of Particles is the result of large-scale scientific collaboration, spanning continents and decades. In his famous lectures on physics, Richard Feynman touched on the goal of physics: ``...the aim is to see complete nature as different aspects of one set of phenomena. That is the problem in basic theoretical physics today---to find the laws behind experiment; to amalgamate these classes."\cite{feynman} The Standard Model is the result of such amalgamations, beginning with the unification of two of nature's fundamental forces: electromagnetism and the weak interaction. At high-enough energies---about \eGeV{100}, corresponding to a temperature of \SI{e15}{\kelvin}\cite{electroweak}---the two forces become indistinguishable, forming the so-called electroweak interaction. Further unification of the fundamental forces has yet to be achieved, but the incorporation of the Higgs mechanism---which gives mass to elementary particles---and the understanding of the strong interaction has lead to the Standard Model that we know today.

		Thirty-seven elementary particles are described by the Standard Model. They are elementary in the sense that they are not composed of more fundamental constituents. These particles consist of twelve quarks, twelve leptons, and thirteen force-carrying bosons.

		
		%% SUBSECTION : BOSONS
		\subsection{Bosons}
			\label{sec:bosons}
			Bosons are particles that obey Bose-Einstein statistics. Some bosons are elementary particles. These are the six force carriers: $\gamma$ (electromagnetic), $g$ (strong), $W^{\pm}$ \& $Z$ (weak), and $H$ (Higgs). The bosons have associated with them no conserved quantities and are not constrained by nearby particles in the states they can occupy. The bosons are also distinct in that the force carriers mediate interactions between the other particles. Two of the force carriers are massless, the photon and the gluon. The weak force carriers in contrast are quite heavy, weighing in at $\sim \nolinebreak \mGeV{90}$. This mass asymmetry was cause for concern during the development of the electroweak theory. To account for it, the Higgs field was proposed as a mechanism to imbue the $W$ and $Z$ bosons with mass.

		
		%% SUBSECTION : FERMIONS
		\subsection{Fermions}
			\label{sec:fermions}

			Leptons and quarks constitute fermions. The leptons are the electron, muon, tau particles and their associated neutrinos. The quarks are the up, down, charm, strange, top, and bottom. All of these particles have antiparticles, which are also fermions. Fermions are designated as such because they follow the laws of Fermi-Dirac statistics, the other pillar of the dichotomy that describes systems of particles. That is, identical fermions obey the Pauli exclusion principle. 

			These elementary fermions---leptons and quarks---are categorized into three ordered generations (\TAB \ref{table:leptons}), each generation being greater in mass than the one before it. There are two conserved quantities associated with fermions: The lepton quantum number and the baryon quantum number. Like momentum or energy, any process involving fermions must conserve these quantities. 

			\begin{table}[H]
				\centering
				\captionsetup{width=4in}
				\caption[The Elementary Fermions]{The elementary fermions are divided into generations. Mass increases from generation one to three. The first generation is the most stable. Every particle in this table has an associated anti-particle, which is also a fermion.}
				\label{table:leptons}
				\begin{tabular}{cccc}
					\toprule
					 & First & Second & Third \\
					 & Generation & Generation & Generation \\
					\midrule
					\multirow{2}{*}{\emph{Leptons}} & \HepParticle{\Pelectron} & \HepParticle{\Pmu} & $\tau$ \\ 
					 & \HepParticle{\Pnue} & \HepParticle{\Pnum} & $\nu_{\tau}$ \\
					\midrule
					\multirow{2}{*}{\emph{Quarks}} & \HepParticle{\Pup}{}{} & \HepParticle{\Pcharm}{}{} & \HepParticle{\Ptop}{}{} \\
					 & \HepParticle{\Pdown}{}{} & \HepParticle{\Pstrange}{}{} & \HepParticle{\Pbottom}{}{} \\
					\bottomrule
				\end{tabular}
			\end{table}

			\begin{table}[H]
				\centering
				\caption[Summary of Particle Discoveries]{\bf A summary of particle discoveries. \rm There are distinct error bars on each of the reported masses that I have not included here, but the values in this table are current as of 2012. Particle masses are taken from the Particle Data Group \cite{PDG}, except for that of the Proton, Antiproton, and Neutron, which come from another source \cite{proton_mass} and are current as of 2010.}
				\label{table:particleList}
					\begin{tabular}{lcccl}
						\toprule
						Particle Name & Symbol & Mass & Year & Credit \\
						\midrule
						Photon 				& \HepParticle{\Pphoton} 		& 0 				& --- 		& --- \\
						Electron 			& \HepParticle{\Pelectron}		& \mMeV{0.511} 		& 1897 		& J.J. Thomson \\
						Proton 				& \HepParticle{\Pproton}		& \mMeV{938.3}		& 1919		& Ernest Rutherford \\
						Neutron 			& \HepParticle{\Pneutron}		& \mMeV{939.6}		& 1932		& James Chadwick \\
						Positron			& \HepParticle{\APelectron}		& \mMeV{0.511}		& 1932		& Carl D. Anderson \\
						Muon				& \HepParticle{\Pmu}			& \mMeV{105.7} 		& 1937 		& Seth Neddermeyer, et al. \\
						Pion 				& \HepParticle{\Ppi}			& \mMeV{135.0}		& 1947		& C. F. Powell, et al. \\
						Kaon 				& \HepParticle{\PK}				& \mMeV{497.6}		& 1947		& George Dixon, et al. Rochester \\
						Lambda Baryon		& $\Lambda^0$					& \mMeV{1116}		& 1950		& V D Hopper, et al. \\
						Antiproton			& \HepParticle{\APproton}		& \mMeV{938.3}		& 1955		& Owen Chamberlain, et al. \\
						Electron Neutrino 	& \HepParticle{\Pnue}			& $<$ \mmeV{2}		& 1956		& Frederick Reines \& Clyde Cowan \\
						Muon Neutrino 		& \HepParticle{\Pnum}			& $<$ \mmeV{190}	& 1962		& Leon Lederman, et al. \\
						Xi Baryon 			& $\Xi^0$						& \mMeV{1315} 		& 1964		& Brookhaven National Laboratory \\
						Up Quark 		 	& u 							& \mMeV{2.3}		& 1969 		& SLAC \\
						Down Quark 		 	& d  							& \mMeV{4.8}		& 1969 		& SLAC \\
						Strange Quark 		& \HepParticle{\Pstrange}		& \mMeV{95}			& 1969 		& SLAC \\
						J$/ \psi$ Meson		& \HepParticle{\PJpsi}			& \mMeV{3097}		& 1974		& Burton Richter, et al. \\
						Charm Quark 		& \HepParticle{\Pcharm} 		& \mGeV{1.28}		& 1974 		& Burton Richter, et al. \\
						Tau 				& \HepParticle{\Ptau}			& \mMeV{1777}		& 1975		& Martin Perl, et al. \\
						Upsilon Meson 		& $\Upsilon$					& \mMeV{9460} 		& 1977		& Fermilab \\
						Bottom Quark 		& \HepParticle{\Pbottom}		& \mGeV{4.18} 		& 1977 		& Fermilab \\
						Gluon				& \HepParticle{\Pgluon}			& 0 				& 1979		& DESY \\
						W Boson 			& W$^{\pm}$ 					& \mGeV{80.39}		& 1983		& Carlo Rubbia, et al. \\
						Z Boson 			& Z$^0$ 						& \mGeV{91.19}		& 1983		& Carlo Rubbia, et al. \\
						Top Quark 			& \HepParticle{\Ptop} 			& \mGeV{173.1}		& 1995		& Fermilab \\
						Tau Neutrino 		& \HepParticle{\Pnut}			& $<$ \mMeV{18.2}	& 2000		& Fermilab \\
						Higgs Boson 		& \HepParticle{\PHiggs} 		& \mGeV{125.9}		& 2012		& CERN (ATLAS) \\				
						\bottomrule
					\end{tabular}
			\end{table}

			\vspace{0.3in}
			\newpage

	
	%% SECTION : THE NEUTRINO AND ITS PROPERTIES
	\section{The Neutrino and its Properties}

		It was in 1930 that Wolfgang Pauli sought to solve the problem that surrounded $\beta$-decay. The problem was that the process in which a neutron decayed into a proton and an electron seemed to violate the well-established conservation laws of energy, momentum, and spin. Pauli hypothesized the existence of a small neutral particle to reconcile this discrepancy. The idea was a bold one, since the proton and electron were the only two known particles at the time. Neils Bohr opposed Pauli's hypothesis and preferred a statistical explanation that allowed for the violation of conservation laws. Pauli's postulated particle later became known as the neutrino\footnote{More specifically, the electron neutrino. The existence of the other two flavors---the muon neutrino and tau neutrino---were not hypothesized until 1948 and 1974, respectively.}, meaning \emph{little neutral one}.

		Its existence was confirmed in a famous experiment by Fredrick Reines and Clyde Cowan in 1956\cite{first_nu_detection}. They used a mixture of water and cadmium chloride as an interaction medium for electron antineutrinos, which invoke an inverse $\beta$-decay process (\EQ \ref{eq:inv_beta_decay}).
		\begin{equation}
			\label{eq:inv_beta_decay}
			\HepProcess{\HepParticle{\APnue}{}{} + \HepParticle{\Pproton}{}{} \HepTo \HepParticle{\Pneutron}{}{} + \HepParticle{\Ppositron}{}{}}
		\end{equation}
		The resultant positron quickly annihilates with a nearby electron, producing two photons. On a larger time scale, the resultant neutron eventually captures on one of the cadmium nuclei, exciting it. When the nucleus de-excites, it emits a photon. Reines and Cowan measured a time delay between detection of the pair-annihilation photons and the neutron capture photon, confirming the neutrino's existence. \\

		In 1962, the muon neutrino was discovered by shooting pi mesons from a synchrotron towards a steel wall\cite{Danby1962}. The pi mesons decay into a muon and a muon neutrino. The former were stopped at the wall while the latter continue through to a detector that photographed the byproducts of a neutrino interaction with aluminum. Much later, in 2000, the tau neutrino was finally discovered at Fermilab\cite{tau}. 

		Of all the particles found in nature, neutrinos are perhaps the most ubiquitous. They are produced by a variety of physical processes: radioactive decays, particle interactions, nuclear fusion, and stellar explosions to name a few. Neutrinos are not charged or very massive (see \TAB \ref{table:particleList}), so they only interact very weakly and can have very long lifetimes. Earth is being bathed in a sea of neutrinos streaming out from the nuclear process at the center of the sun. The solar neutrino flux at Earth is $\sim \SI{6e10}{\per\square\centi\metre\per\second}$\cite{solar_flux}.

		Like all leptons, neutrinos interact via the weak force. That is, every neutrino interaction is moderated by an exchange of a $W$ or $Z$ boson. These interactions can be classified into two groups: charged current (CC) and neutral current (NC). Examples of these interactions are presented in the Feynman diagrams below:

		\begin{minipage}{0.45\textwidth}
			\begin{figure}[H]
				\centering
					\begin{tikzpicture}[
				        thick,
				        % Set the overall layout of the tree
				        level/.style={level distance=1.5cm},
				        level 2/.style={sibling distance=2.6cm},
				        level 3/.style={sibling distance=2cm}
					    ]
					    \coordinate
					        child[grow=left]{
					            child {
					                node {$e^-$}
					                % The 'edge from parent' is actually not needed because it is
					                % implicitly added.
					                edge from parent [fermion-out]
					            }
					            child {
					                node {$\nu_e$}
					                edge from parent [fermion-in]
					            }
					            edge from parent [photon] node [above=3pt] {$W$}
					        }
					        % I have to insert a dummy child to get the tree to grow
					        % correctly to the right.
					        child[grow=right, level distance=0pt] {
					        child  {
					            node {$n$}
					            edge from parent [fermion-in]
					        }
					        child {
					            node {$p$}
					            edge from parent [fermion-out]
					        }
					    };

					\end{tikzpicture}
				\caption[Charged Current Interaction]{\bf Charged current interaction. \rm A $W$ boson is exchanged between the neutron and neutrino to produce an electron and proton.}
				\label{fig:feynman}
			\end{figure}
		\end{minipage}
		\hspace{0.02\textwidth}
		\begin{minipage}{0.45\textwidth}
			\begin{figure}[H]
				\centering
					\begin{tikzpicture}[
				        thick,
				        % Set the overall layout of the tree
				        level/.style={level distance=1.5cm},
				        level 2/.style={sibling distance=2.6cm},
				        level 3/.style={sibling distance=2cm}
					    ]
					    \coordinate
					        child[grow=left]{
					            child {
					                node {$\nu_e$}
					                % The 'edge from parent' is actually not needed because it is
					                % implicitly added.
					                edge from parent [fermion-out]
					            }
					            child {
					                node {$\nu_e$}
					                edge from parent [fermion-in]
					            }
					            edge from parent [photon] node [above=3pt] {$Z$}
					        }
					        % I have to insert a dummy child to get the tree to grow
					        % correctly to the right.
					        child[grow=right, level distance=0pt] {
					        child  {
					            node {$e^-$}
					            edge from parent [fermion-in]
					        }
					        child {
					            node {$e^-$}
					            edge from parent [fermion-out]
					        }
					    };

					\end{tikzpicture}
				\caption[Neutral Current Interaction]{\bf Neutral current interaction. \rm The neutrino and the electron scatter off one another through the exchange of a $Z$ boson.}
				\label{fig:feynman2}
			\end{figure}
		\end{minipage}

		\vspace{0.2in}

		Neutrinos, without interacting, can change their flavor. For example, what is measured as an electron neutrino now might be measured as a muon neutrino later. This is the phenomenon known as ``neutrino oscillation.'' It was predicted by Bruno Pontecorvo in 1957\cite{pont} and happens because the neutrino flavor eigenstates do not correspond to the mass eigenstates. The two eigenstates are related by a mixing matrix, $U_{\alpha k}$:
		\begin{equation}
			\Ket{\nu_\alpha} = \sum_{k=1}^{N} U_{\alpha k} \Ket{\nu_k}
		\end{equation}
		Where $N$ is the number of possible neutrino flavors (currently thought to be three).

		For oscillations in a vacuum, the probability of a neutrino of flavor $\alpha$ oscillating to one of flavor $\beta$ depends on the angles of the mixing matrix, the distance traveled, the energy of the neutrino, and the mass square difference between mass states\cite{PDG}.
		\begin{equation}
			P(\nu_\alpha \rightarrow \nu_\beta) = \delta_{\alpha\beta} - 4 \sum_{j>i} U_{\alpha i} U_{\beta i}^* U_{\alpha j}^* U_{\beta j} \sin^2\left(\frac{1.27 \ \Delta m_{ij}^2 L}{E}\right)
		\end{equation}

		Differences in the oscillation probabilities appear when neutrinos travel in matter. This is known as the MSW (Mikheyev-Smirnov-Wolfenstein) effect\cite{mkw}. It comes about due to the coherent scattering of electron neutrinos on atomic electrons. Muon and tau neutrinos are not affected because there are rarely muons or tau particles to scatter them. The mass eigenstates then propagate differently than they otherwise would in a vacuum. 
%-----------------------------------------------------------------------------
%-----------------------------------------------------------------------------