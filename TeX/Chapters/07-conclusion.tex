%-----------------------------------------------------------------------------
%
%          PHYSICS  M.S. THESIS
%          JUSTIN A. VASEL
%
%          This began as the template offered by the University of Minnesota, 
%          but I've made a few changes here and there...  
%
%          -->  TeX/Chapters/07-conclusion.tex
%
%-----------------------------------------------------------------------------


\chapter{Conclusion and Discussion}
	\label{conclusion_chapter}
	\vspace{-0.2in}

	\begin{quoting}
		\noindent \large ``It is a capital mistake to theorize before one has data. Insensibly one begins to twist facts to suit theories, instead of theories to suit facts." \normalsize

		--- Sir Arthur Conan Doyle
	\end{quoting}

	\chapterIntro{S}{upernovae are not a well-understood phenomenon.} Various neutrino observatories have joined a network to provide advance warning of these rare events by exploiting the immediate release of neutrinos produced within their collapsing cores. The Helium and Lead Observatory is the world's first dedicated supernova detector and will be joining the Supernova Early Warning System once it is fully operational.

	HALO is designed to be a low-maintenance, low-cost, high-uptime detector. ``Astronomically patient'' is the name of the game; the expected occurrence rate of galactic supernovae is two or three per century, so HALO is being designed to last for decades. The detector is fully constructed and has been taking data since May 2011. However, HALO is not yet ready to fulfill its destiny. Designing an experiment that can span decades requires very thoughtful planning and careful execution. Furthermore, an experiment that will likely have but one opportunity to perform its only duty must be capable of doing so at all times. There must be redundancy and fail-safes and continuous monitoring. 

	These considerations are currently the forefront of progress being made on HALO. Within the coming months, calibration of the detector will begin by exposing it to the well-known neutron source, californium. This will provide a deeper understanding of the detector's response and will help strengthen the Monte Carlo simulation. Also the effort to separate the background gamma ray sources from the neutrino-induced neutron signal will soon be underway. The water shielding in front of the detector is not yet in place, but that will not be the case for long now that the \he counters are believed to be in their final positions. Finally, the software that listens for the supernova neutrino signal will be written and implemented, connecting HALO to SNEWS and marking the beginning of what is likely to be a long but worthwhile tenure as the world's supernova watchdog. 

	It has been very rewarding to be a part of the HALO collaboration. And I am grateful to have played a meaningful role in the history of this experiment. My work mainly centered around providing a long-lasting infrastructure for the experiment. I developed a networking scheme that is simple, scalable, and redundant which has allowed the experiment to start placing networked devices into their final configurations. I also improved the monitoring capabilities of the experiment. I organized the complex interconnectedness of hardware into a tabular hardware map for easy referencing. Most recently, I built the basic framework for what will be a powerful remote monitoring tool. The framework is simple and modular, which will allow the collaboration to make modifications or improvements as the need arises. In the coming weeks, I will continue to develop the remote monitoring application on top of this framework. Soon, it will be ready to use, and HALO will be one step closer to full operation.


%-----------------------------------------------------------------------------
%-----------------------------------------------------------------------------