%-----------------------------------------------------------------------------
%
%          PHYSICS  M.S.     THESIS
%          JUSTIN A. VASEL
%
%          This began as the template offered by the University of Minnesota, 
%          but I've made a few changes here and there...  
%
%          -->  supernovae.tex
%
%-----------------------------------------------------------------------------


\chapter{Neutrino Production in Supernovae}
	\label{supernovae_chapter}

	\vspace{-0.2in}

	\begin{quoting}
		\noindent \large ``Stars are phoenixes, rising from their own ashes." \normalsize

		--- Carl Sagan
	\end{quoting}

	\chapterIntro{L}{ittle is known about the behavior of a star moments from death.} The mechanisms of stellar evolution are thought to be fairly well-understood thanks to a robust nuclear theory and computational models. Likewise the after-effects of a stellar explosion are at the very least well-documented through both galactic and extra-galactic observation. But, the moments immediately before and immediately following these eruptions are poorly understood. 
	
	\section{Supernova Taxonomy}
		Supernovae are classified merely by their spectra. In that respect, supernovae are to astronomers as insects are to biologists, minerals to geologists, or artifacts to archeologists. Supernovae are catalogued into groups based solely on their appearance. It is---as Ernest Rutherford would call it---stamp collecting. 

		Generally speaking there are two types of supernova: Type I and Type II. The difference between them is that the spectra of Type II supernovae contain very prominant hydrogen lines, while those of Type I do not. Furthermore the Type I supernovae are divided into two types depending on whether their spectra contain silicon (Type Ia) or not. The latter of these---devoid of hydrogen and devoid of silicon---are again divided in two, depending on whether they're helium rich (Type Ib) or helium poor (Type Ic). 

		Type Ia supernovae are perhaps the most interesting of the three Type I supernovae. The situation in which Type Ia supernovae occur is thought to be that of a white dwarf in binary orbit with a companion star. Matter from the companion accretes onto the white dwarf until the Chandrasekhar limit of \ 1.4 \emph{M}$_\odot$ is reached. At that point the degeneracy pressure is overcome by gravity, causing the white dwarf to collapse and then explode when its carbon ignites. The explosion is so catastrophic that there is likely no remnant left in its wake. 

		What's interesting about Type Ia supernovae is that they all explode at the same mass, 1.4 \emph{M}$_\odot$. The consistency of their erruptions---and, thus, their spectra---make them ideal standard candles, allowing us to calculate intergalactic distances simply by measuring the peak brightness of their light curve.

		The Type II supernovae are also divided into two groups and are distinguished by the shape of their light curve. Type II-L has a light curve that decays rather linearly after maximum brightness and Type II-P has one that plateaus after maximum brightness. Stars that become Type II supernovae are massive. As a result, they tend to live relatively short lives, making them more likely to be found in the arms of spiral galaxies where active star formation is occuring. Their large mass will eventually lead to a collapse of the star's core that will blow the outer layers of the star away and leave behind a neutron star or a black hole.

		These supernovae will be the focus of the rest of the chapter and are the motivation behind this thesis and the HALO experiment. Throughout the rest of this thesis, I will make no further distinction between the Type II-L and Type II-P supernovae, but will instead refer to them collectively as ``core-collapse'' supernovae.
	\section{Shell Burning}
		\filler
	\section{Core Collapse}
		\filler
	\section{Rebound}
		\filler
	\section{Remnants}
		\filler


%-----------------------------------------------------------------------------
%-----------------------------------------------------------------------------