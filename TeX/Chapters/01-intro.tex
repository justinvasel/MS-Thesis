%-----------------------------------------------------------------------------
%
%          PHYSICS  M.S. THESIS
%          JUSTIN A. VASEL
%
%          This began as the template offered by the University of Minnesota, 
%          but I've made a few changes here and there...  
%
%          -->  TeX/Chapters/01-intro.tex
%
%-----------------------------------------------------------------------------


\chapter{Introduction}
	\label{intro_chapter}
	\vspace{-0.2in}

	\begin{quoting}
		\noindent \large ``Measure what is measurable, and make measurable what is not so." \normalsize

		--- Galileo Galilei
	\end{quoting}

	\chapterIntro{G}{one are the days of the solitary experimental physicist.} As far as particle physics is concerned, that is. The experiments of today demand more than gold foil or cathode ray tubes. Today's experiments require high energy, big machinery, scrupulous analysis, and a generous amount of funding. Particle physics has entered the age of collaborations, the amassing of dozens, sometimes hundreds or thousands, of brilliant minds for a single experiment. 

	Such experiments require careful planning and years of preparation before they can even begin. The Helium and Lead Observatory (HALO) in Ontario, Canada is an experiment in the late stages of preparation. HALO will exploit the prompt neutrino signal produced in the explosion of massive stars to predict the appearance of the explosion's delayed optical signal. 

	I have had the honor of contributing to the progress of this experiment during the past year. HALO is not yet fully operational. The detector has been constructed and is continually taking data, but there are still several administrative tasks that must be completed before HALO is ready to receive the next supernova signal. In this thesis, I will detail the current status of HALO and my contributions to the experiment. I will explain the relevant physics behind HALO's scientific goals and I will discuss what remains to be done before HALO is in its final configuration.

	\vspace{0.3in}
	\noindent
	\emph{Chapter summary:}

	\begin{description}
		\item[Chapter 2] A primer on neutrinos, with an emphasis on topics that are most relevant to production in supernovae and interactions with nucleons.
		\item[Chapter 3] A detailed summary of our current understanding of core-collapse supernovae as it relates to neutrino production. 
		\item[Chapter 4] An explanation of the Supernova Early Warning System, of which HALO will be a member once fully operational.
		\item[Chapter 5] A description of the HALO experiment.
		\item[Chapter 6] A presentation of my contributions to HALO and what is left to do before HALO is fully operational.
 	\end{description}



%-----------------------------------------------------------------------------
%-----------------------------------------------------------------------------
