%-----------------------------------------------------------------------------
%
%          PHYSICS  M.S.     THESIS
%          JUSTIN A. VASEL
%
%          This began as the template offered by the University of Minnesota, 
%          but I've made a few changes here and there...  
%
%          -->  app_glossary.tex
%
%-----------------------------------------------------------------------------


\chapter{Glossary and Acronyms}
\label{app_glossary}

\scaps{Care has been taken in this thesis} to minimize the use of jargon and
acronyms. It is my personal philosophy that ideas should be expressed in the 
simplest way possible, avoiding the convoluted use of jargon whenever possible. 
Unfortunately, this cannot always be achieved.  This appendix defines jargon 
terms in a glossary, and contains a table of acronyms and their meaning.


%-------------------------------------------------------------------------------
%    //  G L O S S A R Y
%===============================================================================
\section{Glossary}
\label{jargonapp}
%-------------------------------------------------------------------------------
\begin{itemize}

\item \textbf{Lepton} -- A class of fermionic particles, including the electron, muon, tau and their associated neutrinos and antiparticles.

\end{itemize}


%-------------------------------------------------------------------------------
%    //  A C R O N Y M S
%===============================================================================
\section{Acronyms}
\label{acronymsec}
%-------------------------------------------------------------------------------

%\setlength\LTleft{0pt}
%\setlength\LTright{0pt}

\begin{longtable}{|p{0.25\textwidth}|p{0.75\textwidth}|}
\caption{Acronyms} \label{Acronyms} \\

\hline
Acronym & Meaning \\
\hline \hline
\endfirsthead

\multicolumn{2}{l}%
{{\bfseries \tablename\ \thetable{} -- continued from previous page}} \\
\hline
Acronym & Meaning \\
\hline \hline
\endhead

\hline \hline \multicolumn{2}{|r|}{{Continued on next page}} \\ \hline
\endfoot

\hline \hline
\endlastfoot

HALO & Helium And Lead Observatory \\
SNEWS & SuperNova Early Warning System \\
PMNS & Pontecorvo–Maki–Nakagawa–Sakata 

\end{longtable}


%-------------------------------------------------------------------------------
%-------------------------------------------------------------------------------
