%-----------------------------------------------------------------------------
%
%          PHYSICS  M.S.     THESIS
%          JUSTIN A. VASEL
%
%          This began as the template offered by the University of Minnesota, 
%          but I've made a few changes here and there...  
%
%          -->  my_definitions.tex
%
%-----------------------------------------------------------------------------

%% MATHS
\newcommand{\COS}{{\mathrm{cos}}}
\newcommand{\SIN}{{\mathrm{sin}}}


%% UNITS
% energy (prefix: e)
% \newcommand{\eGeV}[1]{\unit{#1}{\GeV}}
% \newcommand{\eMeV}[1]{\unit{#1}{\MeV}}
% \newcommand{\ekeV}[1]{\unit{#1}{\keV}}
% \newcommand{\eeV}[1]{\unit{#1}{\eV}}

% % mass (prefix: m)
% \newcommand{\mGeV}[1]{\unit{#1}{\GeVovercsq}}
% \newcommand{\mMeV}[1]{\unit{#1}{\MeVovercsq}}
% \newcommand{\mkeV}[1]{\unit{#1}{\keVovercsq}}
% \newcommand{\mmeV}[1]{\unit{#1}{\eVovercsq}}


% energy (prefix: e)
\newcommand{\eGeV}[1]{\SI{#1}{\GeV}}
\newcommand{\eMeV}[1]{\SI{#1}{\MeV}}
\newcommand{\ekeV}[1]{\SI{#1}{\keV}}
\newcommand{\eeV}[1]{\SI{#1}{\eV}}

% mass (prefix: m)
\newcommand{\mGeV}[1]{\SI[per=slash,eVcorrb=0.4ex]{#1}{\giga\eVperc\squared}}
\newcommand{\mMeV}[1]{\SI[per=slash,eVcorrb=0.4ex]{#1}{\mega\eVperc\squared}}
\newcommand{\mkeV}[1]{\SI[per=slash,eVcorrb=0.4ex]{#1}{\kilo\eVperc\squared}}
\newcommand{\mmeV}[1]{\SI[per=slash,eVcorrb=0.4ex]{#1}{\eVperc\squared}}


%% HEADINGS
\newcommand{\CONT}{\noindent}
\newcommand{\FIG}{Fig. \nolinebreak }
\newcommand{\FIGS}{Figs. \nolinebreak }
\newcommand{\SEC}{Sec.\ }
\newcommand{\SECS}{Secs.\ }
\newcommand{\TAB}{Table }
\newcommand{\TABS}{Tables }
\newcommand{\EQ}{Eq.\nolinebreak \ }
\newcommand{\EQS}{Eqs.\nolinebreak \ }
\newcommand{\APP}{Appendix }
\newcommand{\APPS}{Appendices }
\newcommand{\CHP}{Chapter }
\newcommand{\CHPS}{Chapters }


%% OTHER
\def\imagetop#1{\vtop{\null\hbox{#1}}}

\newcommand{\plan}[1] {
	\noindent {\color{BrickRed} [#1]} \\
}

\newcommand{\filler} {
	{\color{MidnightBlue} \lipsum}
}

\newcommand{\scaps}[1] {
	\minion \textsc{#1}\normalfont
}

\newcommand{\chapterIntro}[2] {
	\noindent
	\lettrine[lines=4,nindent=4pt]{#1}{\scaps{#2}}
}

\definecolor{light-gray}{gray}{0.3}

\newcommand{\he}{
	$^3$He
}

\newcommand{\confirm}[1]{
	{\color{Rhodamine} #1}
}

% place an inline node that is remembered for tikz
% \tikzremember{<node name>}
%   note: you have to compile twice
\newcommand{\tikzremember}[1]{{
  \tikz[remember picture,overlay]{\node (#1) at (0,11pt) { };}
}}

\colorlet{punct}{red!60!black}
\definecolor{background}{HTML}{EEEEEE}
\definecolor{delim}{RGB}{20,105,176}
\colorlet{numb}{magenta!60!black}

\lstdefinelanguage{json}{
    basicstyle=\footnotesize\ttfamily,
    stepnumber=1,
    numbersep=8pt,
    showstringspaces=false,
    extendedchars=\true,
    frame=lines,
    backgroundcolor=\color{background},
    literate=
     *{0}{{{\color{numb}0}}}{1}
      {1}{{{\color{numb}1}}}{1}
      {2}{{{\color{numb}2}}}{1}
      {3}{{{\color{numb}3}}}{1}
      {4}{{{\color{numb}4}}}{1}
      {5}{{{\color{numb}5}}}{1}
      {6}{{{\color{numb}6}}}{1}
      {7}{{{\color{numb}7}}}{1}
      {8}{{{\color{numb}8}}}{1}
      {9}{{{\color{numb}9}}}{1}
      {:}{{{\color{punct}{:}}}}{1}
      {,}{{{\color{punct}{,}}}}{1}
      {\{}{{{\color{delim}{\{}}}}{1}
      {\}}{{{\color{delim}{\}}}}}{1}
      {[}{{{\color{delim}{[}}}}{1}
      {]}{{{\color{delim}{]}}}}{1},
}

\lstdefinelanguage{object-c}{
    basicstyle=\footnotesize\ttfamily,
    stepnumber=1,
    numbersep=8pt,
    showstringspaces=false,
    extendedchars=\true,
    frame=lines,
    backgroundcolor=\color{background},
    literate=
     *{0}{{{\color{numb}0}}}{1}
      {1}{{{\color{numb}1}}}{1}
      {2}{{{\color{numb}2}}}{1}
      {3}{{{\color{numb}3}}}{1}
      {4}{{{\color{numb}4}}}{1}
      {5}{{{\color{numb}5}}}{1}
      {6}{{{\color{numb}6}}}{1}
      {7}{{{\color{numb}7}}}{1}
      {8}{{{\color{numb}8}}}{1}
      {9}{{{\color{numb}9}}}{1}
      {:}{{{\color{punct}{:}}}}{1}
      {,}{{{\color{punct}{,}}}}{1}
      {=}{{{\color{punct}{=}}}}{1}
      {"}{{{\color{punct}{"}}}}{1}
      {@}{{{\color{punct}{@}}}}{1}
      {\{}{{{\color{delim}{\{}}}}{1}
      {\}}{{{\color{delim}{\}}}}}{1}
      {[}{{{\color{delim}{[}}}}{1}
      {]}{{{\color{delim}{]}}}}{1}
      {)}{{{\color{delim}{)}}}}{1}
      {(}{{{\color{delim}{(}}}}{1},
}


%-----------------------------------------------------------------------------
%-----------------------------------------------------------------------------